\section{Utenti}

Il sito si rivolge a tutti gli utenti desiderosi di ricevere delle informazioni o di provare uno degli sport fluviali offerti, e usa un linguaggio piuttosto semplice e accattivante per destare l'interesse. L'utenza prevista � da ricercarsi principalmente in zone vicine alla Valle del Brenta.\\
Per allargare il pi� possibile il bacino d'utenza, si � cercato di rendere il sito semplice, accessibile e facile da utilizzare su pi� dispositivi e browser possibili. \\ \\
Nella fase di progettazione sono stati individuati tre categorie di utente:
\paragraph{Utente non registrato}\mbox{}\\
L'utente non registrato pu� accedere alle prime 5 pagine presentate nella \hyperlink{sez3}{Sezione 2}. Nella pagina \texttt{Attivit�} visualizza un'elenco completo delle varie attivit� offerte (con il relativo prezzo), senza per� la possibilit� di effettuare prenotazioni.

\paragraph{Utente autenticato}\mbox{}\\
L'utente che ha effettuato il login ha la possibilit� di  prenotare le attivit� nell'apposita pagina e pu� accedere alla pagina \hyperlink{pannelloUtente}{Pannello Utente} con la quale pu� interagire per svolgere varie azioni, illustrate nella \hyperlink{sez3}{Sezione 2}.

\paragraph{Amministratore}\mbox{}\\
L'utente amministratore rappresenta la segreteria del centro Onda Selvaggia, essa dunque deve poter gestire le prenotazioni, le attivit� offerte e gli utenti. Per fare ci� � stata realizzata un pagina apposita, \hyperlink{pannelloAdmin}{Pannello Admin}, nella quale � possibile effettuare l'attivit� di amministrazione (dettagli nella \hyperlink{sez3}{Sezione 2}).