\section{Struttura del sito}
Il sito si compone di nelle seguenti pagine.
\paragraph{Home}\mbox{}\\
La pagina Home racchiude tutte le informazioni principali riguardanti l'azienda Onda Selvaggia.
\hypertarget{paginaAttivita}{\paragraph{Attivit�}}\mbox{}\\
E' una delle pagine principali poich� attraverso questa pagina, dopo aver effettuato il login, � possibile poter prenotare un'attivita in base al giorno e al numero di partecipanti. Inoltre � una pagina dinamica che viene modificata dal pagina \textbf{Pannello Admin}.
\paragraph{Contattaci}\mbox{}\\
Contiene tutte le informazioni necessarie a poter raggiungere e contattare il centro fluviale Onda Selvaggia.
\paragraph{Login e Registrati}\mbox{}\\
Sono le pagine di registrazione al sito e di login. Il login � necessario per poter prenotare le attivit�. Inoltre gli utenti loggati possono accedere alla loro pagina personale \textbf{Pannello utente}.
\paragraph{Pannello Utente}\mbox{}\\
Questa pagina permette agli utenti di poter visualizzare le prenotazioni attive e il loro stato (pagata o non pagata). Per prenotazioni attive si intendono le prenotazioni di attivit� che devono essere svolte e l'utente ha la possibilt� di poterle eliminare. Inoltre viene visualizzato lo storico delle prenotazioni e in relazione ad esse l'utente pu� esprimere un giudizio sull'attivit� svolta.
Infine nel pannello � possibile gestire i dati anagrafici e i dati dell'account e l'eliminazione di esso.


\paragraph{Pannello Admin}\mbox{}\\
Il pannello di amministrazione prensenta varie schede le quali ricoprono specifiche funzionalit�:


\subparagraph{Attivit�:} 
\begin{identazione}
	In questa scheda � possibile creare, modificare ed eliminare macroattivit� ed attivit�.
	Creazione, modifica ed eliminazione si riflettono nella pagina \hyperlink{paginaAttivita}{Attivit�} dove vengono inserite, modificate od eliminate le macroattivit� o attivit� offerte nella stessa.
\end{identazione}

\subparagraph{Statistiche:} 
\begin{identazione}
	Vengono visualizzate statistiche sulle attivit� (prenotazioni e valutazione media) e sugli utenti.
\end{identazione}

\subparagraph{Utenti:}
\begin{identazione}
	In questa scheda vengono visualizzate le informazioni dei vari account degli utenti. ********Da continuare*******
\end{identazione}

\subparagraph{Prenotazioni:}
\begin{identazione}
	[ Testo da rivedere... In questa scheda] si possono visualizzare le prenotazioni in sospeso, cio� prenotazioni di attivit� da svolgere ed � possibile confermare il pagamento o meno. In mancanza di possibilit� di poter effettuare pagamenti online (ad esempio PayPal) e quindi non potendo automatizzare la gestione dei pagamenti  si � deciso di ovviare al problema attraverso un metodo rudimentale ma efficace. Inoltre � possibile convalidare le prenotazioni (cio� prenotazioni pagate e svolte) attraverso il codice QR (che viene scannerizzato attraverso la webcam) o il codice alfanumerico presente nel pdf della prenotazione che il cliente presenta il giorno dell'attivit� da svolgere.
	Infine � possibile  visualizzare lo storico di tutte le prenotazioni effettuate da tutti gli utenti iscritti al sito.
\end{identazione}

\subparagraph{Disponibilit�:}
\begin{identazione}
	contenuto...
\end{identazione}